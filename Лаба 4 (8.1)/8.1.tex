\documentclass[%
 reprint,
%superscriptaddress,
%groupedaddress,
%unsortedaddress,
%runinaddress,
%frontmatterverbose, 
%preprint,
%preprintnumbers,
%nofootinbib,
%nobibnotes,
%bibnotes,
 amsmath,amssymb,
 aps,
%pra,
%prb,
%rmp,
%prstab,
%prstper,
%floatfix,
]{revtex4-2}
\usepackage{multirow}
\usepackage{graphicx}% Include figure files
\usepackage{dcolumn}% Align table columns on decimal point
\usepackage{bm}% bold math
%\usepackage{hyperref}% add hypertext capabilities
%\usepackage[mathlines]{lineno}% Enable numbering of text and display math
%\linenumbers\relax % Commence numbering lines

%\usepackage[showframe,%Uncomment any one of the following lines to test 
%%scale=0.7, marginratio={1:1, 2:3}, ignoreall,% default settings
%%text={7in,10in},centering,
%%margin=1.5in,
%%total={6.5in,8.75in}, top=1.2in, left=0.9in, includefoot,
%%height=10in,a5paper,hmargin={3cm,0.8in},
%]{geometry}
\usepackage[utf8x]{inputenc} % Включаем поддержку UTF8  
\usepackage[russian]{babel}  % Включаем пакет для поддержки русского языка 
\usepackage[normalem]{ulem}  % для зачекивания текста

\usepackage[noend]{algorithmic}
\def\algorithmicrequire{\textbf{Вход:}}
\def\algorithmicensure{\textbf{Выход:}}
\def\algorithmicif{\textbf{если}}
\def\algorithmicthen{\textbf{то}}
\def\algorithmicelse{\textbf{иначе}}
\def\algorithmicelsif{\textbf{иначе если}}
\def\algorithmicfor{\textbf{для}}
\def\algorithmicforall{\textbf{для всех}}
\def\algorithmicdo{}
\def\algorithmicwhile{\textbf{пока}}
\def\algorithmicrepeat{\textbf{повторять}}
\def\algorithmicuntil{\textbf{пока}}
\def\algorithmicloop{\textbf{цикл}}
% переопределение стиля комментариев
\def\algorithmiccomment#1{\quad// {\sl #1}}

\usepackage{caption}
\usepackage{subcaption}
\usepackage{multirow}
\usepackage[table,xcdraw]{xcolor}
\begin{document}



\title{Лабораторная работа 8.1\\Определение постоянных Стефана Больцмана и Планка из анализа теплового излучения накаленного тела}% Force line breaks with \\



\author{Васиильев Михаил Владимирович}
\affiliation{%
 Студент 3 курса РТ\\
}%

\collaboration{Московский физико-технический институт}%\noaffiliation

\date{6 октября 2021 г.}% It is always \today, today,
             %  but any date may be explicitly specified
             

\begin{abstract}
При помощи модели АЧТ проводятся измерения температуры оптическим пирометром с исчезающей нитью и термопарой, исследуются излучение накаленных тел с различной испускательной способностью, определяются постоянные Планка и Стефана-Больцмана.
\begin{description}
\item[Оборудование]
Пирометр, неоновая лампочка, модель АЧТ, разогретые кольца.
\end{description}
\end{abstract}

%\keywords{Suggested keywords}%Use showkeys class option if keyword
                              %display desired
\maketitle

%\tableofcontents

\section{Теоретическая часть.}
Для измерения температуры разогретых тел, удаленных от наблюдателя, применяют методы оптической пирометрии, основанные на использовании зависимости испускательной способности исследуемого тела от температуры.

Под яркостной температурой понимают температуру абсолютно черного тела, при которой его спектральная испускательная способность равна спектральной испускательной способности исследуемого тела при той же длине волны. Именно эту температуру мы будем измерять в данной работе.

Яркостная температура тела всегда ниже его термодинамической температуры. Это связано с тем, что любое нечерное тело излучает меньше, чем абсолютно черное тело при той же температуре. Чтобы получить величину термодинамической температуры тела, надо вводить дополнительные поправки, которые определяются для каждого материала экспериментально.

\section{Экспериментальная установка и методика}

Экспериментальная установка (рис. 3) состоит из оптического пирометра 9, модели абсолютно черного тела (АЧТ), трех исследуемых образцов (18, 19, 20), блока питания (1) и цифровых вольтметров В7-22А и В7-38.

\begin{figure}[h]
\includegraphics[scale = 0.3]{S.png}
\caption{Схема экспериментальной установки: 1 -- блок питания; 2 -- тумблер включения питания пирометра и образцов; 3 -- тумблер нагрева нити пирометра; 4 -- кнопка нагрев нити; 5 -- кнопка охлаждение нити; 6 -- тумблер переключения образцов;7 -- регулятор мощности нагрева образцов; 8 -- окуляр пирометра; 9 -- корпус пирометра; 10 -- объектив пирометра; 11 --переключение диапазонов: 700 -- 1200 C$ ^o $ -- вниз, 1200 -- 2000 C $ ^o $ вверх; 12 -- ручка перемещения красного светофильтра; 13 -- регулировочный винт; 14 -- вольтметр (напряжение на лампе накаливания); 15 -- амперметр (ток через образцы); 16 -- вольтметр в цепи термопары; 17 -- модель АЧТ; 18 -- трубка с кольцами из материалов с разной излучательной способностью; 19 -- лампа накаливания; 20 -- неоновая лампочка}
\end{figure}

Измерение яркостной температуры раскаленного тела производится при помощи оптического пирометра с исчезающей нитью, основанного на визуальном сравнении яркости раскаленной нити с яркостью изображения исследуемого тела. Равенство видимых яркостей, наблюдаемых через монохроматический светофильтр, фиксируется по исчезновению изображения нити на фоне раскаленного тела.
Яркостный метод измерения температуры основан, в соответствии с формулой Планка, на зависимости испускательной способности абсолютно черного тела от температуры и длины волны.

Оптический пирометр представляет собой зрительную трубу, внутри которой имеется накаливаемая нить, расположенная в плоскости изображения исследуемого раскаленного тела, а также темно-красный светофильтр. Через окуляр одновременно наблюдается изображение исследуемого тела и раскаленной нити.
Если в том узком спектральном интервале, который пропускается светофильтром, яркость нити меньше яркости раскаленного тела, то нить видится темной полоской на светлом фоне, и наоборот. При совпадении яркостей нить перестает быть видимой на фоне изображения раскаленного тела. Регулировка яркости нити осуществляется изменением тока, протекающего через нее.

Шкалу прибора, измеряющего ток через нить, предварительно градуируют по абсолютно черному телу, термодинамическую температуру которого измеряют с помощью термопары.

Модель АЧТ представляет собой керамическую трубку диаметром $3$ мм и длиной $50$ мм, закрытую с одного конца и окруженную для теплоизоляции внешним кожухом. Нагрев трубки осуществляется намотанной на ней нихромовой спиралью, питаемой от источника тока. Полость трубки и особенно ее дно излучают практически как абсолютно черное тело. Температура модели АЧТ измеряется хромель-алюмелевой термопарой, один спай которой вмонтирован в дно трубки, а другой находится при комнатной температуре на клемме цифрового вольтметра В7-38, измеряющего ЭДС термопары.

\section{Основные результаты и их обсуждение}

Потенциированная зависимость Стефана-Больцмана имеет вид:
%%\begin{figure}[h]
%%	\includegraphics[scale=0.8]{1.png}
%%\end{figure}

Из обобщенного МНК следует, что $n = 3.38 \pm 0.5$, $\sigma = (4,36 \pm 0.6) \cdot 10^{-7} \text{Вт} \cdot \text{М}^{-2} \cdot \text{К}^{-4} $ при $T = 1323$ , что не соответствует теории. Все остальные измеренные и помереные величины сведены в таблицу: 



\section{Заключение}

Была проверена степенная зависимость Стефана-Больцмана, она оказалась в эксперименте такой, как в теории. Получены близкие к теоретическим значения постоянной Планка и Стефана-Больцмана.

\section{Данные и график}

\begin{table}[h]
\begin{tabular}{|l|l|l|l|l|l|l|l|l|l|}
\hline
№  & Tя   & Tт   & dTт & I mA & V mV & W вт & lnW   & lnT  & dWвт \\ \hline
1  & 871  & 929  & 23  & 594  & 2366 & 1,41 & 0,340 & 6,83 & 0,03 \\ \hline
2  & 885  & 944  & 24  & 651  & 2922 & 1,90 & 0,643 & 6,85 & 0,04 \\ \hline
3  & 901  & 961  & 24  & 670  & 3124 & 2,09 & 0,739 & 6,87 & 0,04 \\ \hline
4  & 950  & 1013 & 25  & 694  & 3373 & 2,34 & 0,851 & 6,92 & 0,05 \\ \hline
5  & 982  & 1047 & 26  & 720  & 3623 & 2,61 & 0,959 & 6,95 & 0,05 \\ \hline
6  & 1017 & 1085 & 27  & 755  & 4006 & 3,02 & 1,107 & 6,99 & 0,06 \\ \hline
7  & 1032 & 1101 & 28  & 772  & 4200 & 3,24 & 1,176 & 7,00 & 0,06 \\ \hline
8  & 1069 & 1140 & 29  & 798  & 4476 & 3,57 & 1,273 & 7,04 & 0,07 \\ \hline
9  & 1147 & 1223 & 31  & 835  & 4924 & 4,11 & 1,414 & 7,11 & 0,08 \\ \hline
10 & 1207 & 1287 & 32  & 898  & 5668 & 5,09 & 1,627 & 7,16 & 0,10 \\ \hline
11 & 1323 & 1411 & 35  & 997  & 6949 & 6,93 & 1,936 & 7,25 & 0,14 \\ \hline
\end{tabular}
\end{table}

\begin{figure}[h]
\includegraphics[scale = 0.45]{G.jpg}
\caption{График зависимости ln(W) от ln(T)}
\end{figure}

\end{document}