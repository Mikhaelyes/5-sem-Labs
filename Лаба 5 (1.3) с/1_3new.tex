\documentclass[%
 reprint,
%superscriptaddress,
%groupedaddress,
%unsortedaddress,
%runinaddress,
%frontmatterverbose, 
%preprint,
%preprintnumbers,
%nofootinbib,
%nobibnotes,
%bibnotes,
 amsmath,amssymb,
 aps,
%pra,
%prb,
%rmp,
%prstab,
%prstper,
%floatfix,
]{revtex4-2}
\usepackage{multirow}
\usepackage{graphicx}% Include figure files
\usepackage{dcolumn}% Align table columns on decimal point
\usepackage{bm}% bold math
%\usepackage{hyperref}% add hypertext capabilities
%\usepackage[mathlines]{lineno}% Enable numbering of text and display math
%\linenumbers\relax % Commence numbering lines

%\usepackage[showframe,%Uncomment any one of the following lines to test 
%%scale=0.7, marginratio={1:1, 2:3}, ignoreall,% default settings
%%text={7in,10in},centering,
%%margin=1.5in,
%%total={6.5in,8.75in}, top=1.2in, left=0.9in, includefoot,
%%height=10in,a5paper,hmargin={3cm,0.8in},
%]{geometry}
\usepackage[utf8x]{inputenc} % Включаем поддержку UTF8  
\usepackage[russian]{babel}  % Включаем пакет для поддержки русского языка 
\usepackage[normalem]{ulem}  % для зачекивания текста

\usepackage[noend]{algorithmic}
\def\algorithmicrequire{\textbf{Вход:}}
\def\algorithmicensure{\textbf{Выход:}}
\def\algorithmicif{\textbf{если}}
\def\algorithmicthen{\textbf{то}}
\def\algorithmicelse{\textbf{иначе}}
\def\algorithmicelsif{\textbf{иначе если}}
\def\algorithmicfor{\textbf{для}}
\def\algorithmicforall{\textbf{для всех}}
\def\algorithmicdo{}
\def\algorithmicwhile{\textbf{пока}}
\def\algorithmicrepeat{\textbf{повторять}}
\def\algorithmicuntil{\textbf{пока}}
\def\algorithmicloop{\textbf{цикл}}
% переопределение стиля комментариев
\def\algorithmiccomment#1{\quad// {\sl #1}}
\usepackage{caption}

\usepackage{subcaption}
\usepackage{multirow}
\usepackage[table,xcdraw]{xcolor}
\begin{document}



\title{Лабораторная работа 1.3\\Изучение рассеяния медленных электронов на атомах (эффект Рамзауэра)}% Force line breaks with \\



\author{Васильев Михаил Владимирович}
\affiliation{%
 Студент 3 курса РТ\\
}%

\collaboration{Московский физико-технический институт}%\noaffiliation

\date{\today}% It is always \today, today,
             %  but any date may be explicitly specified
             

\begin{abstract}
Исследуется энергетическая зависимость вероятности рассеяния электронов атомами ксенона, определяются энергии электронов, при которых наблюдается "просветление" ксенона, и оценивается размер его внешней электронной оболочки.  
\begin{description}
\item[Оборудование]
Тиратрон ТГ3-01/1.3Б, осциллограф, стабилизированный БНС.
\end{description}
\end{abstract}

%\keywords{Suggested keywords}%Use showkeys class option if keyword
                              %display desired
\maketitle

%\tableofcontents

\section{Теоретическая часть.}

Вводится понятие эффективного сечения реакции $\sigma = \frac{N}{nv}$, характеризующая вероятность перехода системы из двух сталкивающихся частич в определенное состояние, в результате их рассеяния. Знаменатель равен плотности потока всех рассеиваемых частиц, числитель - число таких переходов.
К. Рамзауэр исследовал зависимость поперечных сечений упрогого рассеяния электронов (с энергией до 10 ЭВ) на атомах аргона. В результате этих исследований было обнаружено явление, получившее название \textit{эффекта Рамзауэра}.

С точки зрения квантовой теории атом по отношению к электронной волне ведет себя как преломляющая среда с относительным показателем преломления
\begin{equation*}
n = \frac{\lambda}{\lambda^\prime} = \sqrt{1-\frac{U}{E}},
\end{equation*}
где $U$, $E$ -- соответственно потенциальная и полная энергии электрона внутри атома.

Будем считать, что электрон рассеивается на одномерной прямоугольной потенциальной яме конечной глубины. Такая модель является хорошим приближением для атомов тяжелых инертных газов, отличающихся наиболее компактной структурой и резкой внешней границей. Решение задачи о прохождении частицы с энергией $E$ над потенциальной ямой шириной $l$ и глубиной $U_0$ не составит труда найти из уравнения Шредингера:\\
\begin{equation*}
\psi^{\prime\prime}+k^2\psi=0, \ \text{где}\
k^2 =\begin{cases}
2mE/\hbar^2 & x<0, x>l\\
(2mE+U_0)/\hbar^2 & 0<x<l
\end{cases}.
\end{equation*}
Коэффициент прохождения равен отношению квадратов амплитуд прошедшей и падающей волн и определяется выражением:
\begin{equation*}
\frac{1}{D} = 1 + \frac{U_0^2}{4E(E+U_0)}\sin^2(k_2l).
\end{equation*}
Минимум последнего выражения отвечает квантовому аналогу просветления оптики, так как при выполнении условия
\begin{equation*}
\tag{$\star$}
\label{eq:uslovie}
\sqrt{\frac{2m(E+U_0)}{\hbar^2}}l = \pi n, \ n\in\mathbb{N},
\end{equation*}
коэффициент прохождения частицы над ямой становится равным единице, то есть достигает своего максимального значения.
Отметим, что условие~(\ref{eq:uslovie}) легко получить, рассматривая интерференцию электронов волн де Бройля в атоме:\\
\begin{itemize}
	\item
	Условие первого интерференционного максимума:
	\begin{equation}
	\label{eq:1}
	2l = \frac{h}{\sqrt{2m(E_1+U_0)}}.
	\end{equation}
	\item
	Условие первого интерференционного минимума:
	\begin{equation}
	\label{eq:2}
	2l =\frac{3}{2} \frac{h}{\sqrt{2m(E_1+U_0)}}.
	\end{equation}			
\end{itemize}

Решая совместно уравнения~(\ref{eq:1}, \ref{eq:2}) можно получить:
\begin{equation}
\label{eq:l}
l = \frac{h\sqrt{5}}{\sqrt{32m(E_2-E_1)}}.
\end{equation}
Понятно, что энергии $E_1$ и $E_2$ соответствуют энергиям электронов, прошедших разность потенциалов $V_1$ и $V_2$, то есть $E_1 = eV_1$ и $E_2 = eV_2$. 

По измеренным величинам $E_1$ и $E_2$, используя формулы~(\ref{eq:1}, \ref{eq:2}), можно рассчитать эффективную глубину потенциальной ямы атома:
\begin{equation}
\label{eq:U_0}
U_0 = \frac{4}{5}E_2 - \frac{9}{5}E_1
\end{equation}

Согласно квантовой механике зависимость вероятности рассеяния электрона от его энергии можно определить из соотношения:
\begin{equation}
\label{eq:w}
w(U) = -\frac{1}{C}\ln \frac{I(U)}{I_0},
\end{equation}
где $I_0$ -- ток катода, а $C$ -- некторая постоянная.

\section{Экспериментальная установка}
Принципиальная схема установки для изучения эффекта Рамзауэра приведена на рис.1. На лампу Л подаётся синусоидальное напряжение частоты 50 Гц от источника питания Г34, БНС - стабилизованый блок накала лампы; исследуемый сигнал подаётся на электронный осциллограф (ЭО); цифрами обозначены номера ножек лампы. 

Реально на экране ЭО удаётся надёжно наблюдать лишь один минимум в сечении рассеивания электронов и следующий за ним максимум. Дело в том, что уже при  $n = 2$ происходит пробой. 

Схема экспериментальной установки, изображённой на рис.3 в нашей работе конструктивно осуществлена следующим образом. Лампа-тиратрон ТГЗ-01/1.3Б, заполнена инертным газом, расположена на БИП. Напряжение к электродам лампы подаётся от источников питания, находящихся на корпусе прибора. Регулировка напряжения и выбор режима работы установки производиться при помощи ручек управления на рис.2.


\begin{figure}[h!]
\includegraphics[scale = 0.45]{3.png}
\caption{Схема включения тиратрона (Л), модулирующего переменного напряжения и осциллогра а (ЭО) для изучения эффекта Рамзауэра. БНС -- стабилизированный блок накала электрода; цифрами обозначены номера ножек лампы}
\end{figure}

\begin{figure}[h!]
\includegraphics[scale = 0.35]{4.png}
\caption{Блок-схема экспериментальной установки}
\end{figure}

В данной работе для изучения эффекта Рамзауэра используется тиратрон Т 3-01/1.3Б, заполненный инертным газом. Схематическое изображение тиратрона и его конструкция приведены на рис. 3. Электроны, эмитируемые катодом тиратрона, ускоряются напряжением V, приложенным между катодом и ближайшей к нему сеткой. Затем электроны рассеиваются на атомах инертного газа (ксенона). Все сетки 1, 2, 3 соединены между собой и имеют одинаковый потенциал, примерно равный потенциалу анода 6. Поэтому между первой сеткой 1 и анодом практически нет поля. Рассеянные электроны отклоняются в сторону и уходят на сетку, а оставшаяся часть электронов достигает анода и создает анодный ток Iа. Таким образом, поток электронов N(x) на расстоянии x от ускоряющей сетки (т.е. число электронов, проходящих через поперечное сечение лампы в точке в единицу времени) уменьшается с ростом x от начального значения $N_0$ у катода (в точке $x = 0$) до некоторого значения $N_a$ у анода (в точке $x = L$).
\begin{figure}[h!]
\includegraphics[scale = 0.45]{5.png}
\caption{Схематическое изображение тиратрона (слева) и его конструкция (справа): 1, 2, 3 -- сетки; 4 -- внешний металлический цилиндр; 5 -- катод; 6 -- анод; 7 -- накаливаемая спираль}
\end{figure}


\section{Методика измерений}
В эксперименте будем исследовать ВАХ двумя методами: статическими и динамическим. В статическом методе будем снимать показания напряжения с вольтметра и амперметра при различных значениях напряжения анода, в динамическом методе сразу используем картину на осциллографе. 


\section{Основные результаты и их обсуждение.}
Имеем 2 таблицы основным результатов.
По данным строятся вольт-амперные характеристики для синего $U_\text{н} = 2.63$ В и красного $U_\text{н} = 2.89$ В и зависимость вероятности рассеивания от напряжения. Определяем по напряжению пробоя $U_\text{п} = 12.1$ В, что лампа заполнена ксеноном. Определим глубину потенциальной ямы $U_\text{я} = 2.64 \pm 0,05$ эВ.

\newpage
\section{Заключение.}
Были доказаны доказаны квантовые свойства света на эффекте Рамзауэра, простроены ВАХ и зависимость вероятности рассеивания от напряжения, был определён инертный газ в лампе. Выводы хорошо согласуются с теорией. 


\begin{figure}[h!]
\includegraphics[scale=0.5]{1.jpg}
\caption{ВАХ при Uн = 2.63 В  (красный) и при Uн = 2.89 В (синий)}
\end{figure}

\newpage
$ $
\newpage

\begin{figure}[]
\includegraphics[scale=0.5]{2.jpg}
\caption{Зависимость вероятности рассеивания электрона от напряжения анода $\omega$ при Uн = 2.63 В  (красный) и при Uн = 2.89 В (синий)}
\end{figure}

\newpage
% Please add the following required packages to your document preamble:
% \usepackage{multirow}
\begin{table}[]
\begin{tabular}{llllll|l|l|l|l|l|}
\cline{1-5} \cline{7-11}
\multicolumn{5}{|c|}{\multirow{2}{*}{ВАХ и $\omega$ при Uн = 2.63 В}}                                                                                            &  & \multicolumn{5}{c|}{\multirow{2}{*}{ВАХ и $\omega$ при   Uн = 2.89 В}} \\
\multicolumn{5}{|c|}{}                                                                                                                                  &  & \multicolumn{5}{c|}{}                                         \\ \cline{1-5} \cline{7-11} 
\multicolumn{1}{|l|}{Vi, мВ} & \multicolumn{1}{l|}{V, В}  & \multicolumn{1}{l|}{I, мкА} & \multicolumn{1}{l|}{$\omega$}        & \multicolumn{1}{l|}{$\Delta\omega$}       &  & Vi, мВ     & V, В     & I, мкА    & $\omega$           & $\Delta\omega$          \\ \cline{1-5} \cline{7-11} 
\multicolumn{1}{|l|}{0}      & \multicolumn{1}{l|}{0,32}  & \multicolumn{1}{l|}{0}      & \multicolumn{1}{l|}{1}        & \multicolumn{1}{l|}{0}        &  & 0          & 0,3      & 0         & 1           & 0           \\ \cline{1-5} \cline{7-11} 
\multicolumn{1}{|l|}{0}      & \multicolumn{1}{l|}{0,6}   & \multicolumn{1}{l|}{0}      & \multicolumn{1}{l|}{1}        & \multicolumn{1}{l|}{0}        &  & 0          & 0,6      & 0         & 1           & 0           \\ \cline{1-5} \cline{7-11} 
\multicolumn{1}{|l|}{0}      & \multicolumn{1}{l|}{0,9}   & \multicolumn{1}{l|}{0}      & \multicolumn{1}{l|}{1}        & \multicolumn{1}{l|}{0}        &  & 0          & 0,9      & 0         & 1           & 0           \\ \cline{1-5} \cline{7-11} 
\multicolumn{1}{|l|}{0}      & \multicolumn{1}{l|}{1,2}   & \multicolumn{1}{l|}{0}      & \multicolumn{1}{l|}{1}        & \multicolumn{1}{l|}{0}        &  & 0          & 1,2      & 0         & 1           & 0           \\ \cline{1-5} \cline{7-11} 
\multicolumn{1}{|l|}{0}      & \multicolumn{1}{l|}{1,5}   & \multicolumn{1}{l|}{0}      & \multicolumn{1}{l|}{1}        & \multicolumn{1}{l|}{0}        &  & 0          & 1,5      & 0         & 1           & 0           \\ \cline{1-5} \cline{7-11} 
\multicolumn{1}{|l|}{0}      & \multicolumn{1}{l|}{1,8}   & \multicolumn{1}{l|}{0}      & \multicolumn{1}{l|}{1}        & \multicolumn{1}{l|}{0}        &  & 0          & 1,8      & 0         & 1           & 0           \\ \cline{1-5} \cline{7-11} 
\multicolumn{1}{|l|}{0}      & \multicolumn{1}{l|}{2,2}   & \multicolumn{1}{l|}{0}      & \multicolumn{1}{l|}{1}        & \multicolumn{1}{l|}{0}        &  & 0          & 2,12     & 0         & 1           & 0           \\ \cline{1-5} \cline{7-11} 
\multicolumn{1}{|l|}{0}      & \multicolumn{1}{l|}{2,29}  & \multicolumn{1}{l|}{0}      & \multicolumn{1}{l|}{1}        & \multicolumn{1}{l|}{0}        &  & 0,1        & 2,21     & 1         & 0,999494    & 0,029633    \\ \cline{1-5} \cline{7-11} 
\multicolumn{1}{|l|}{0,53}   & \multicolumn{1}{l|}{2,42}  & \multicolumn{1}{l|}{5,3}    & \multicolumn{1}{l|}{0,992639} & \multicolumn{1}{l|}{0,029633} &  & 0,7        & 2,33     & 7         & 0,703312    & 0,019382    \\ \cline{1-5} \cline{7-11} 
\multicolumn{1}{|l|}{1,4}    & \multicolumn{1}{l|}{2,51}  & \multicolumn{1}{l|}{14}     & \multicolumn{1}{l|}{0,765688} & \multicolumn{1}{l|}{0,019382} &  & 2,5        & 2,44     & 25        & 0,509558    & 0,009944    \\ \cline{1-5} \cline{7-11} 
\multicolumn{1}{|l|}{4,4}    & \multicolumn{1}{l|}{2,6}   & \multicolumn{1}{l|}{44}     & \multicolumn{1}{l|}{0,498134} & \multicolumn{1}{l|}{0,009944} &  & 5,8        & 2,51     & 58        & 0,381466    & 0,004888    \\ \cline{1-5} \cline{7-11} 
\multicolumn{1}{|l|}{10,1}   & \multicolumn{1}{l|}{2,71}  & \multicolumn{1}{l|}{101}    & \multicolumn{1}{l|}{0,303991} & \multicolumn{1}{l|}{0,004888} &  & 12,7       & 2,62     & 127       & 0,262174    & 0,002395    \\ \cline{1-5} \cline{7-11} 
\multicolumn{1}{|l|}{17,4}   & \multicolumn{1}{l|}{2,79}  & \multicolumn{1}{l|}{174}    & \multicolumn{1}{l|}{0,176903} & \multicolumn{1}{l|}{0,002395} &  & 21,6       & 2,73     & 216       & 0,181339    & 0,002033    \\ \cline{1-5} \cline{7-11} 
\multicolumn{1}{|l|}{19,1}   & \multicolumn{1}{l|}{2,82}  & \multicolumn{1}{l|}{191}    & \multicolumn{1}{l|}{0,155124} & \multicolumn{1}{l|}{0,002033} &  & 29,2       & 2,81     & 292       & 0,135452    & 0,00091     \\ \cline{1-5} \cline{7-11} 
\multicolumn{1}{|l|}{26,5}   & \multicolumn{1}{l|}{2,91}  & \multicolumn{1}{l|}{265}    & \multicolumn{1}{l|}{0,078615} & \multicolumn{1}{l|}{0,00091}  &  & 38,8       & 2,91     & 388       & 0,092187    & 0,00025     \\ \cline{1-5} \cline{7-11} 
\multicolumn{1}{|l|}{33,5}   & \multicolumn{1}{l|}{3,02}  & \multicolumn{1}{l|}{335}    & \multicolumn{1}{l|}{0,023848} & \multicolumn{1}{l|}{0,00025}  &  & 48,3       & 3,04     & 483       & 0,058852    & 0           \\ \cline{1-5} \cline{7-11} 
\multicolumn{1}{|l|}{37,1}   & \multicolumn{1}{l|}{3,11}  & \multicolumn{1}{l|}{371}    & \multicolumn{1}{l|}{0}        & \multicolumn{1}{l|}{0}        &  & 53,6       & 3,23     & 536       & 0,043004    & -0,00011    \\ \cline{1-5} \cline{7-11} 
\multicolumn{1}{|l|}{39}     & \multicolumn{1}{l|}{3,2}   & \multicolumn{1}{l|}{390}    & \multicolumn{1}{l|}{-0,01167} & \multicolumn{1}{l|}{-0,00011} &  & 55,6       & 3,51     & 556       & 0,037428    & 0,000564    \\ \cline{1-5} \cline{7-11} 
\multicolumn{1}{|l|}{29,8}   & \multicolumn{1}{l|}{3,3}   & \multicolumn{1}{l|}{298}    & \multicolumn{1}{l|}{0,051194} & \multicolumn{1}{l|}{0,000564} &  & 51,4       & 3,81     & 514       & 0,049383    & 0,00022     \\ \cline{1-5} \cline{7-11} 
\multicolumn{1}{|l|}{33,9}   & \multicolumn{1}{l|}{3,33}  & \multicolumn{1}{l|}{339}    & \multicolumn{1}{l|}{0,021075} & \multicolumn{1}{l|}{0,00022}  &  & 64,9       & 4,02     & 649       & 0,013887    & 0,000242    \\ \cline{1-5} \cline{7-11} 
\multicolumn{1}{|l|}{33,6}   & \multicolumn{1}{l|}{3,61}  & \multicolumn{1}{l|}{336}    & \multicolumn{1}{l|}{0,023152} & \multicolumn{1}{l|}{0,000242} &  & 65,7       & 4,1      & 657       & 0,012023    & 0,000296    \\ \cline{1-5} \cline{7-11} 
\multicolumn{1}{|l|}{32,9}   & \multicolumn{1}{l|}{3,92}  & \multicolumn{1}{l|}{329}    & \multicolumn{1}{l|}{0,028071} & \multicolumn{1}{l|}{0,000296} &  & 66,6       & 4,21     & 666       & 0,009952    & 0,000345    \\ \cline{1-5} \cline{7-11} 
\multicolumn{1}{|l|}{32,3}   & \multicolumn{1}{l|}{4,22}  & \multicolumn{1}{l|}{323}    & \multicolumn{1}{l|}{0,032371} & \multicolumn{1}{l|}{0,000345} &  & 67,3       & 4,31     & 673       & 0,00836     & 0,000491    \\ \cline{1-5} \cline{7-11} 
\multicolumn{1}{|l|}{30,6}   & \multicolumn{1}{l|}{4,52}  & \multicolumn{1}{l|}{306}    & \multicolumn{1}{l|}{0,045004} & \multicolumn{1}{l|}{0,000491} &  & 68,4       & 4,45     & 684       & 0,005893    & 0,000574    \\ \cline{1-5} \cline{7-11} 
\multicolumn{1}{|l|}{29,7}   & \multicolumn{1}{l|}{4,8}   & \multicolumn{1}{l|}{297}    & \multicolumn{1}{l|}{0,051979} & \multicolumn{1}{l|}{0,000574} &  & 69,1       & 4,61     & 691       & 0,004343    & 0,000622    \\ \cline{1-5} \cline{7-11} 
\multicolumn{1}{|l|}{29,2}   & \multicolumn{1}{l|}{5,11}  & \multicolumn{1}{l|}{292}    & \multicolumn{1}{l|}{0,055946} & \multicolumn{1}{l|}{0,000622} &  & 69,5       & 4,71     & 695       & 0,003464    & 0,000632    \\ \cline{1-5} \cline{7-11} 
\multicolumn{1}{|l|}{29,1}   & \multicolumn{1}{l|}{5,41}  & \multicolumn{1}{l|}{291}    & \multicolumn{1}{l|}{0,056747} & \multicolumn{1}{l|}{0,000632} &  & 69,9       & 4,84     & 699       & 0,002591    & 0,000765    \\ \cline{1-5} \cline{7-11} 
\multicolumn{1}{|l|}{27,8}   & \multicolumn{1}{l|}{5,73}  & \multicolumn{1}{l|}{278}    & \multicolumn{1}{l|}{0,067425} & \multicolumn{1}{l|}{0,000765} &  & 70,2       & 4,93     & 702       & 0,001939    & 0,000898    \\ \cline{1-5} \cline{7-11} 
\multicolumn{1}{|l|}{26,6}   & \multicolumn{1}{l|}{6,01}  & \multicolumn{1}{l|}{266}    & \multicolumn{1}{l|}{0,077735} & \multicolumn{1}{l|}{0,000898} &  & 70,5       & 5,01     & 705       & 0,00129     & 0,001042    \\ \cline{1-5} \cline{7-11} 
\multicolumn{1}{|l|}{25,4}   & \multicolumn{1}{l|}{6,31}  & \multicolumn{1}{l|}{254}    & \multicolumn{1}{l|}{0,088521} & \multicolumn{1}{l|}{0,001042} &  & 71,1       & 5,11     & 711       & 0           & 0,001239    \\ \cline{1-5} \cline{7-11} 
\multicolumn{1}{|l|}{23,9}   & \multicolumn{1}{l|}{6,65}  & \multicolumn{1}{l|}{239}    & \multicolumn{1}{l|}{0,102743} & \multicolumn{1}{l|}{0,001239} &  & 70,4       & 5,22     & 704       & 0,001506    & 0,001381    \\ \cline{1-5} \cline{7-11} 
\multicolumn{1}{|l|}{22,9}   & \multicolumn{1}{l|}{6,9}   & \multicolumn{1}{l|}{229}    & \multicolumn{1}{l|}{0,112729} & \multicolumn{1}{l|}{0,001381} &  & 70,1       & 5,33     & 701       & 0,002156    & 0,001616    \\ \cline{1-5} \cline{7-11} 
\multicolumn{1}{|l|}{21,4}   & \multicolumn{1}{l|}{7,21}  & \multicolumn{1}{l|}{214}    & \multicolumn{1}{l|}{0,128557} & \multicolumn{1}{l|}{0,001616} &  & 70,2       & 5,4      & 702       & 0,001939    & 0,001879    \\ \cline{1-5} \cline{7-11} 
\multicolumn{1}{|l|}{19,9}   & \multicolumn{1}{l|}{7,51}  & \multicolumn{1}{l|}{199}    & \multicolumn{1}{l|}{0,145537} & \multicolumn{1}{l|}{0,001879} &  & 70,8       & 5,53     & 708       & 0,000644    & 0,002134    \\ \cline{1-5} \cline{7-11} 
\multicolumn{1}{|l|}{18,6}   & \multicolumn{1}{l|}{7,84}  & \multicolumn{1}{l|}{186}    & \multicolumn{1}{l|}{0,161321} & \multicolumn{1}{l|}{0,002134} &  & 69,5       & 5,75     & 695       & 0,003464    & 0,002388    \\ \cline{1-5} \cline{7-11} 
\multicolumn{1}{|l|}{17,43}  & \multicolumn{1}{l|}{8,13}  & \multicolumn{1}{l|}{174,3}  & \multicolumn{1}{l|}{0,176501} & \multicolumn{1}{l|}{0,002388} &  & 69,4       & 5,84     & 694       & 0,003683    & 0,002635    \\ \cline{1-5} \cline{7-11} 
\multicolumn{1}{|l|}{16,4}   & \multicolumn{1}{l|}{8,44}  & \multicolumn{1}{l|}{164}    & \multicolumn{1}{l|}{0,190733} & \multicolumn{1}{l|}{0,002635} &  & 69,3       & 5,92     & 693       & 0,003903    & 0,002817    \\ \cline{1-5} \cline{7-11} 
\multicolumn{1}{|l|}{15,7}   & \multicolumn{1}{l|}{8,72}  & \multicolumn{1}{l|}{157}    & \multicolumn{1}{l|}{0,200924} & \multicolumn{1}{l|}{0,002817} &  & 45,3       & 6,13     & 453       & 0,068612    & 0,00261     \\ \cline{1-5} \cline{7-11} 
\multicolumn{1}{|l|}{16,5}   & \multicolumn{1}{l|}{9,01}  & \multicolumn{1}{l|}{165}    & \multicolumn{1}{l|}{0,189312} & \multicolumn{1}{l|}{0,00261}  &  & 63,2       & 6,52     & 632       & 0,017927    & 0,002536    \\ \cline{1-5} \cline{7-11} 
\multicolumn{1}{|l|}{16,8}   & \multicolumn{1}{l|}{9,16}  & \multicolumn{1}{l|}{168}    & \multicolumn{1}{l|}{0,185102} & \multicolumn{1}{l|}{0,002536} &  & 61,1       & 6,81     & 611       & 0,023071    & 0,002763    \\ \cline{1-5} \cline{7-11} 
\multicolumn{1}{|l|}{15,9}   & \multicolumn{1}{l|}{9,25}  & \multicolumn{1}{l|}{159}    & \multicolumn{1}{l|}{0,197967} & \multicolumn{1}{l|}{0,002763} &  & 58,3       & 7,11     & 583       & 0,030211    & 0,002817    \\ \cline{1-5} \cline{7-11} 
\multicolumn{1}{|l|}{15,7}   & \multicolumn{1}{l|}{9,32}  & \multicolumn{1}{l|}{157}    & \multicolumn{1}{l|}{0,200924} & \multicolumn{1}{l|}{0,002817} &  & 55,4       & 7,44     & 554       & 0,037977    & 0,002844    \\ \cline{1-5} \cline{7-11} 
\multicolumn{1}{|l|}{15,6}   & \multicolumn{1}{l|}{9,46}  & \multicolumn{1}{l|}{156}    & \multicolumn{1}{l|}{0,202417} & \multicolumn{1}{l|}{0,002844} &  & 52,8       & 7,72     & 528       & 0,045293    & 0,002898    \\ \cline{1-5} \cline{7-11} 
\multicolumn{1}{|l|}{15,4}   & \multicolumn{1}{l|}{9,59}  & \multicolumn{1}{l|}{154}    & \multicolumn{1}{l|}{0,205432} & \multicolumn{1}{l|}{0,002898} &  & 49,9       & 8,03     & 499       & 0,053891    & 0,002926    \\ \cline{1-5} \cline{7-11} 
\multicolumn{1}{|l|}{15,3}   & \multicolumn{1}{l|}{9,69}  & \multicolumn{1}{l|}{153}    & \multicolumn{1}{l|}{0,206954} & \multicolumn{1}{l|}{0,002926} &  & 47,4       & 8,32     & 474       & 0,061715    & 0,003011    \\ \cline{1-5} \cline{7-11} 
\multicolumn{1}{|l|}{15}     & \multicolumn{1}{l|}{9,88}  & \multicolumn{1}{l|}{150}    & \multicolumn{1}{l|}{0,211581} & \multicolumn{1}{l|}{0,003011} &  & 45,3       & 8,61     & 453       & 0,068612    & 0,003069    \\ \cline{1-5} \cline{7-11} 
\multicolumn{1}{|l|}{14,8}   & \multicolumn{1}{l|}{10}    & \multicolumn{1}{l|}{148}    & \multicolumn{1}{l|}{0,214717} & \multicolumn{1}{l|}{0,003069} &  & 43,1       & 8,91     & 431       & 0,076189    & 0,003099    \\ \cline{1-5} \cline{7-11} 
\multicolumn{1}{|l|}{14,7}   & \multicolumn{1}{l|}{10,11} & \multicolumn{1}{l|}{147}    & \multicolumn{1}{l|}{0,216301} & \multicolumn{1}{l|}{0,003099} &  & 43,1       & 9,22     & 431       & 0,076189    & 0,003099    \\ \cline{1-5} \cline{7-11} 
\multicolumn{1}{|l|}{14,7}   & \multicolumn{1}{l|}{10,2}  & \multicolumn{1}{l|}{147}    & \multicolumn{1}{l|}{0,216301} & \multicolumn{1}{l|}{0,003099} &  & 40         & 9,54     & 400       & 0,087551    & 0,003099    \\ \cline{1-5} \cline{7-11} 
\multicolumn{1}{|l|}{14,7}   & \multicolumn{1}{l|}{10,34} & \multicolumn{1}{l|}{147}    & \multicolumn{1}{l|}{0,216301} & \multicolumn{1}{l|}{0,003099} &  & 39,4       & 9,81     & 394       & 0,089851    & 0,003069    \\ \cline{1-5} \cline{7-11} 
\multicolumn{1}{|l|}{14,8}   & \multicolumn{1}{l|}{10,41} & \multicolumn{1}{l|}{148}    & \multicolumn{1}{l|}{0,214717} & \multicolumn{1}{l|}{0,003069} &  & 30,9       & 10,05    & 309       & 0,126839    & 0,00304     \\ \cline{1-5} \cline{7-11} 
\multicolumn{1}{|l|}{14,9}   & \multicolumn{1}{l|}{10,54} & \multicolumn{1}{l|}{149}    & \multicolumn{1}{l|}{0,213144} & \multicolumn{1}{l|}{0,00304}  &  & 39,1       & 10,2     & 391       & 0,091014    & 0,003011    \\ \cline{1-5} \cline{7-11} 
\multicolumn{1}{|l|}{15}     & \multicolumn{1}{l|}{10,65} & \multicolumn{1}{l|}{150}    & \multicolumn{1}{l|}{0,211581} & \multicolumn{1}{l|}{0,003011} &  & 39,4       & 10,31    & 394       & 0,089851    & 0,002968    \\ \cline{1-5} \cline{7-11} 
\multicolumn{1}{|l|}{15,15}  & \multicolumn{1}{l|}{10,76} & \multicolumn{1}{l|}{151,5}  & \multicolumn{1}{l|}{0,209256} & \multicolumn{1}{l|}{0,002968} &  & 39,6       & 10,43    & 396       & 0,08908     & 0,002954    \\ \cline{1-5} \cline{7-11} 
\multicolumn{1}{|l|}{15,2}   & \multicolumn{1}{l|}{10,85} & \multicolumn{1}{l|}{152}    & \multicolumn{1}{l|}{0,208486} & \multicolumn{1}{l|}{0,002954} &  & 39,9       & 10,51    & 399       & 0,087932    & 0,003378    \\ \cline{1-5} \cline{7-11} 
\multicolumn{1}{|l|}{13,8}   & \multicolumn{1}{l|}{11,14} & \multicolumn{1}{l|}{138}    & \multicolumn{1}{l|}{0,231063} & \multicolumn{1}{l|}{0,003378} &  & 40,9       & 10,61    & 409       & 0,084164    & 0,003282    \\ \cline{1-5} \cline{7-11} 
\multicolumn{1}{|l|}{14,1}   & \multicolumn{1}{l|}{11,41} & \multicolumn{1}{l|}{141}    & \multicolumn{1}{l|}{0,226038} & \multicolumn{1}{l|}{0,003282} &  & 40,9       & 10,71    & 409       & 0,084164    & 0,003135    \\ \cline{1-5} \cline{7-11} 
\multicolumn{1}{|l|}{14,58}  & \multicolumn{1}{l|}{11,7}  & \multicolumn{1}{l|}{145,8}  & \multicolumn{1}{l|}{0,218216} & \multicolumn{1}{l|}{0,003135} &  & 41,2       & 10,83    & 412       & 0,083052    & 0,002983    \\ \cline{1-5} \cline{7-11} 
\multicolumn{1}{|l|}{15,1}   & \multicolumn{1}{l|}{12,1}  & \multicolumn{1}{l|}{151}    & \multicolumn{1}{l|}{0,210029} & \multicolumn{1}{l|}{0,002983} &  & 41,4       & 10,96    & 414       & 0,082315    & 0           \\ \cline{1-5} \cline{7-11} 
                             &                            &                             &                               &                               &  & 39,4       & 11,07    & 394       & 0,089851    & 0           \\ \cline{7-11} 
                             &                            &                             &                               &                               &  & 41,4       & 11,54    & 414       & 0,082315    & 0           \\ \cline{7-11} 
\end{tabular}
\end{table}

\end{document}