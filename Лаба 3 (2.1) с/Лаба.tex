%класс документа, А4, шрифт
\documentclass[reprint, amsmath, amssymb, aps,]{revtex4-2}

\usepackage[T2A]{fontenc} %кодировка
\usepackage[utf8]{inputenc} %кодировка исходного текста
\usepackage[english, russian]{babel} %локализация и переносы

%математика
\usepackage{amsmath, amsfonts, amssymb, amsthm, mathtools}

%графика
\usepackage{graphicx}
\DeclareGraphicsExtensions{.pdf,.png,.jpg}
\usepackage{wrapfig} %обтекание фигур

\usepackage{multirow}
\usepackage{dcolumn}% Align table columns on decimal point
\usepackage{bm}% bold math

\begin{document}
\title{Лабораторная работа 2.1:\\Опыт Франка-Герца}



\author{Васильев Михаил Владимирович}
\affiliation{%
 Студент 3 курса РТ\\
}%

\collaboration{Московский физико-технический институт}%\noaffiliation

\date{16 сентября 2021 г.}% It is always \today, today,
             %  but any date may be explicitly specified
             

\begin{abstract}
\textbf{В работе исследуются:} энергия первого уровня атома гелия в динамическом и статическом режимах методом электронного возбуждения.
\begin{description}
\item[Оборудование]
Серийная лампа ионизационного манометра ЛМ-2, вольфрамовый катод, источника питания Б74, амперметр, двойная спираль в качестве анода, роль коллектора выполняет металлический цилиндр.
\end{description}
\end{abstract}

\maketitle


\section{Теория}
Одним из простых опытов, подтверждающих существование дискретных уровней энергии атомов, является эксперимент, известный под названием опыта Франка и Герца. Схема опыта изображена на
рис. 1.

Разреженный одноатомный газ (в нашем случае -- гелий) заполняет трехэлектродную лампу. Электроны, испускаемые разогретым катодом, ускоряются в постоянном электрическом поле, созданном между катодом и сетчатым анодом лампы. Передвигаясь от катода к аноду, электроны сталкиваются с атомами гелия. Если энергия электрона, налетающего на атом, недостаточна для того, чтобы перевести его в возбужденное состояние (или ионизовать), то возможны только упругие соударения, при которых электроны почти не теряют энергии, так как их масса в тысячи раз меньше массы атомов. По мере увеличения разности потенциалов между анодом и катодом энергия электронов увеличивается и, в конце концов, оказывается достаточной для возбуждения атомов. При таких -- неупругих -- столкновениях кинетическая энергия налетающего электрона передается одному из атомных электронов, вызывая его переход на свободный энергетический уровень (возбуждение) или совсем отрывая его от атома (ионизация).

Третьим электродом лампы является коллектор. Между ним и анодом поддерживается небольшое задерживающее напряжение (потенциал коллектора меньше потенциала анода). Ток коллектора, пропорциональный числу электронов, попадающих на него за секунду, измеряется микроамперметром.

\begin{figure}[]
\includegraphics[scale = 0.2]{Рис 1}
\caption{Схема опыта Франка и Герца } \label{Рис 1}
\end{figure}

\begin{figure}[]
\includegraphics[scale = 0.2]{Рис 2}
\caption{Схематический вид зависимости тока коллектора от напряжения на аноде} \label{Рис 2}
\end{figure}


При увеличении потенциала анода ток в лампе вначале растет, подобно тому как это происходит в вакуумном диоде (рис. 2). Однако,когда энергия электронов становится достаточной для возбуждения атомов, ток коллектора резко уменьшается. Это происходит потому,что при неупругих соударениях с атомами электроны почти полностью теряют свою энергию и не могут преодолеть задерживающего потенциала (около 1 В) между анодом и коллектором. При дальнейшем увеличении потенциала анода ток коллектора вновь возрастает: электроны, испытавшие неупругие соударения, при дальнейшем движении к аноду успевают набрать энергию, достаточную для преодоления задерживающего потенциала. 

Следующее замедление роста тока происходит в момент, когда часть электронов неупруго сталкивается с атомами два раза: первый раз посередине пути, второй -- у анода и. т. д. Таким образом, на кривой зависимости тока коллектора от напряжения анода имеется ряд максимумов и минимумов, отстоящих друг от друга на равные расстояния $\Delta$ V; эти расстояния равны энергии первого возбужденного состояния.

\newpage
\section{Экспериментальная установка}

\begin{figure}[h]
%\includegraphics[scale = 0.45]{Установка}
\caption{Блок-схема установки для изучения ($\gamma - \gamma$ совпадений). ВСВ -- высоковольтный стабилизированный выпрямитель; С -- сцинтиллятор, кристалл йодистого натрия NaI(Ti); Ф -- формирователь импульсов; СС -- схема совпадений; ПП -- пересчетный прибор} \label{Рис 1}
\end{figure}


\newpage
\section{Методика измерения}


\section{Вывод}


\newpage

\section{Результаты эксперимента}


\newpage
$ $
\newpage
\section{График}

\begin{figure}[h]
\begin{center}

%\includegraphics[width = \textwidth]{1 график}
\caption{График зависимости $N_0$ от $\tau$} \label{Iam}

\end{center}
\end{figure}

\end{document}
